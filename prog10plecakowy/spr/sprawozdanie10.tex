\documentclass[a4paper,11pt]{report}
\usepackage[T1]{fontenc}
\usepackage[utf8]{inputenc}
\usepackage[polish]{babel}
\usepackage{lmodern}
\usepackage{graphicx}
\usepackage{geometry}

\title{Rozwiązanie Problemu Plecakowego z wykorzystaniem metody programowania dynamicznego}
\author{Monika Litwin 200586}
\begin{document}
\maketitle

\begin{figure}
  \begin{center}
  \textbf{Problem plecakowy}
\\
\begin{flushleft}

Jest to jeden z popularnych problemów optymalizacyjnych. Polega on na tym, aby dobrać przedmioty w taki sposób, że ich sumaryczna wartość jest maksymalna, a waga nie większa niż dana pojemność plecaka. Nazywany jest też problemem złodzieja okradającego sklep - musi on załadować jak najbardziej wartościowe przedmioty do plecaka, tak aby się zmieściły, nie może też zabierać ułamkowej części przedmiotów (rozważamy tutaj \textbf{dyskretny problem plecakowy}).

\end{flushleft}
  \end{center}
\end{figure}

\begin{figure}
  \begin{center}
  \textbf{Wykorzystana metoda rozwiązania - programowanie dynamiczne}
\\
\begin{flushleft}

Problem plecakowy w moim algorytmie rozwiązywany jest z użyciem metody programowania dynamicznego. Polega to na tym, że 

\end{flushleft}
  \end{center}
\end{figure}



\end{document}
